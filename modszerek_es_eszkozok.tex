%!TEX root = dolgozat.tex
%%%%%%%%%%%%%%%%%%%%%%%%%%%%%%%%%%%%%%%%%%%%%%%%%%%%%%%%%%%%%%%%%%%%%%%
\chapter{Felhasznált módszerek és eszközök}\label{ch:MODSZEREK_ES_ESZKOZOK}

\begin{osszefoglal}
	Minden nagyobb projekt esetén a fejlesztőknek szükségük van arra, hogy a megfelelő felkészültségük és találékonyságuk mellett, figyelmet fordítsanak a hatékony munkára is. Ehhez szükségük lehet különböző verziókövető rendszerek használatára, projektmenedzsment eszközökre és build rendszerekre.
\end{osszefoglal}


%%%%%%%%%%%%%%%%%%%%%%%%%%%%%%%%%%%%%%%%%%%%%%%%%%%%%%%%%%%%%%%%%%%%%%%%%%%
\section{Verziókövetés}

Napjainkban egyre nagyobb szükség van arra, hogy egy projekt esetén a munka könnyedén megosztható és hordozható legyen a fejlesztők közt. E mellett nagyon fontos a fejlesztési folyamat monitorizálása. Ezen technológiák nélkül szinte elképzelhetetlen a szoftverfejlesztés, úgy csoportos környezetben mint egyedül.

	E célra fejlesztették  ki a verziókövető rendszereket és a projektmenedzsment eszközöket melyek által könnyedén megoszthatóvá válik a fejlesztői munka és folyamatosan ellenőrizhető a fejlesztés folyamata.  
	
	A verziókövető rendszer által  folyamatosan nyomon követhető a projekt fejlődése és ellenőrizhető az egyénenkénti haladás is. Könnyed visszaállítási lehetőséget biztosít arra az esetre, ha valami történne a lokális gépünkön tárolt forrás állományokkal vagy ha bármi hiba történne a fejlesztés során ami visszaállítást igényel. Legnagyobb haszna a verzió követő rendszereknek, az olyan projekteknél van, amelyet több fejlesztő fejleszt egyszerre. Hiszen általában ilyenkor a projekt teljes forrásállománya egy központi tárolóban van elhelyezve ahová mindenki beteszi a változtatásait. Így nagyon egyeszűen követhető, hogy melyik fejlesztő milyen fázisban tart. 
	
	A RoboRun projekt fejlesztése során a forrásállományok tárolására és a fejlesztés nyomkövetésére felhasznált verziókövető rendszer a Git\citep{git}, amely nyílt forráskódú és teljesen ingyenes. Könnyedén megoszthatóak a forrásállományok. A projekt szerzője által használt kliensalkalmazás a  TortoiseGit\citep{tortoisegit}. A TortoiseGit szintén ingyenes szoftver, melyet szükséges telepíteni. Használata egyszerű. A konzol mellett, rendelkezik egy felhasználóbarát grafikus felülettel is, mely még inkább megkönnyíti a használatát.

%%%%%%%%%%%%%%%%%%%%%%%%%%%%%%%%%%%%%%%%%%%%%%%%%%%%%%%%%%%%%%%%%%%%%%%%%%%
\section{Projektmenedzsment}

"A projektmenedzsment az erőforrások szervezésével és azok irányításával foglalkozó szakterület, melynek célja, hogy az erőforrások által végzett munka eredményeként egy adott idő- és költségkereten belül sikeresen teljesüljenek a projekt céljai." (forrás: http://hu.wikipedia.org/wiki/Projektmenedzsment)

	A projektmenedzsment eszközök fő célja tehát, hogy a fejlesztők a specifikáció által meghatározott feladatot adott idő - és költségkereten belül sikeresen tudják teljesíteni. Ennek érdekében számos hasznos funkcionalitást biztosítanak. Ilyen funkcionalitások például, különböző feladatkörök kiosztása, különböző feladatok kiosztása, egy adott folyamatra szánt idő meghatározása, a fejlesztő által eltöltött munkaidő egy adott rész megvalósításával. Mindezek mellett kommunikációs lehetőséget biztosít a fejlesztők között. Lehetőség ad feltölteni dokumentumokat, diagramokat, segédanyagokat a projekthez, ez által megkönnyítve a fejlesztők munkáját. 
	
	A RoboRun projekt fejlesztése során alkalmazott projektmenedzsment eszközként a Redmine\citep{redmine} webes menedzsment eszköz szolgált. A Redmine egy teljesen  nyílt forráskódú és platform független projektmenedzsment rendszer. Egyszerű és letisztult felülete révén könnyen kezelhető. Rengeteg funkcionalítást nyújt, mely nagy segítség lehet a különböző projektek fejlesztése során. A  Redmine által nyújtott néhány fontosabb funkcionalitás: naptár, e-mail értesítés, szerepkör szerinti hozzáférés, wiki és fórum, pluginok engedélyezés, adatbázisok támogatása, stb.



%%%%%%%%%%%%%%%%%%%%%%%%%%%%%%%%%%%%%%%%%%%%%%%%%%%%%%%%%%%%%%%%%%%%%%%%%%%
\section{Build rendszer}

A build rendszereket többnyire projektek menedzselésére és a build folyamat automatizálására alkalmazzák. 

	A RoboRun projekt fejlesztése során alkalmazott build rendszer a Maven\cite{maven}, amelyet Jason van Zyl készített 2002-ben. A Maven egy nyílt forráskódú, platform független eszköz. Leggyakoribb felhasználása a Java nyelvben írt projektek esetében történik. A Maven konfigurációs modellje XML alapú, e mellett bevezetésre került a POM(Project Object Model). A POM az adott projekt szerkezeti vázának teljes leírását tartalmazza és a modulokat azonosítókkal látja el. Tehét a POM egy projekt leírását tartalmazza és a projekthez tartozó összes függőség listáját. Ezen függőségeket a Maven a saját központi tárolójából tölti le a projekt buildelése során. A POM esetén a lépéseket céloknak nevezik. A célok lehetnek előre definiáltak, mint például a forráskód csomagolása és fordítása vagy lehetnek a felhasználó által meghatározott célok.  Mindezt a pom.xml állomány által valósul meg, amely tartalmazza ezeket az információkat. 
	
	A RoboRun projekt esetén a Maven build eszköz legfontosabb szerepe a függőségek, célok és pluginok kielégítése a build folyamat során, hiszen a Maven saját függőség kezelő rendszerrel rendelkezik, amely  a build - elés során letölti a központi tárolóból az előre megadott függőségeket és elhelyezi a lokális tárolóban, ahonnan a jövőben használni fogja. E mellett a Maven lehetőséget nyújt a projekt moduljainak azonosítására a groupID, az artifactID és a verzió szám révén. A groupID logikai csoportokba szervezi a komponenseket, az artifactID minden komponenst egyedi azonosítóval lát el és a verzió az éppen aktuális verziószámot takarja a komponensek esetén.

%%%%%%%%%%%%%%%%%%%%%%%%%%%%%%%%%%%%%%%%%%%%%%%%%%%%%%%%%%%%%%%%%%%%%%%%%%%