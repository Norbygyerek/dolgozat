%!TEX root = dolgozat.tex
%%%%%%%%%%%%%%%%%%%%%%%%%%%%%%%%%%%%%%%%%%%%%%%%%%%%%%%%%%%%%%%%%%%%%%%
\chapter{Következtetés és továbbfejlesztési lehetőségek}\label{ch:KOVETKEZTETESEK}

A RoboRun projekt keretein belül megvalósításra került egy megerősítéses tanulási algoritmusok tesztelésére szolgáló teljes rendszer, amely lehetőséget biztosít a folyamatos és könnyed használatra. A rendszer dinamikus, megvalósítja a komponensek egymástól való elválasztását.A projekthez tartozik egy webes felület, melyen megtekinthetőek a telepített batyuk, az aktív tesztek, illetve a már lefuttatott tesztek eredményei. A rendszer ezen eredményeket egy adatbázisban tárolja. A RoboRun projekt teljesen az OSGi alkalmazás modellre épül.

A RoboRun projekt architektúrája tervezésekor óriási hangsúly volt fektetve a továbbfejleszthetőségre, így a projekt felépítése is ezt tükrözi.

Az OSGi alkalmazás modell által minden komponens külön van választva, így a már meglévő részek rugalmasan továbbfejleszthetőek és kiegészíthetőek. A RoboRun projekt jelen formájában egy OSGi konténerben van telepítve, mely egy GlassFish szerveren fut.

A projekt továbbfejlesztésére számos lehetőség létezik. Egyik legfontosabb ilyen lehetőség, a projekthez egy Eclipse Plugin\cite{eclipseplugin} készítése, mely által az Eclipse fejlesztői környezet, egy olyan környezetet garantálhat, mely megkönnyíti a megerősítéses tanulási algoritmusok implementálását, illetve ez által megvalósítható az, hogy az \texttt{Experiment} komponens a saját gépről futtatható legyen, míg a rendszer többi része egy központi elérésű szerveren található. Ehhez szükség van felhasználni az OSGi által biztosított OSGi Remote Services tulajdonságot. Ez azt jelenti, hogy az OSGi képes távoli metódushívásokra, oly módon, hogy a rendszer egyik része fut egy OSGi konténerben, míg a rendszer többi része  egy teljesen más OSGi konténerben található. Az Eclipse Plugin esetén az Eclipse fejlesztői környezet lenne az egyik OSGi konténer, míg a másik a GlassFish szerverre telepített lenne. Így az erőforrás igényes komponensek, mint például a \texttt{Agent}, \texttt{Environment} és a \texttt{RoboCommnication} komponensek a GlassFish szerveren futnának és az \texttt{Experiment} komponens futna az Eclipse fejlesztői környezetből, amely egy saját lokális számítógépen található. 
Egy másik továbbfejlesztési lehetőségként érdemes megemlíteni az RMI(Remote Method Invocation) kommunikációt. Mely kapcsolatot létesít a szerverrel és képes lekérdezni különböző objektumokat. Ezzel az a probléma, hogy a lekérdezett objektumok szerializálásra kerülnek elküldés előtt és az OSGi szolgáltatásokat nem lehet szerializálni a konténeren kívülre, mert ezen szolgáltatások csak a konténeren belül elérhetőek. Amennyiben erre a problémára sikerülne találni megoldást, ez a továbbfejlesztés nagyon egyszerű és hasznos módja lenne, hiszen ezáltal szintén teljesen áthelyezhető az \texttt{Experiment} komponens saját lokális számítógépre.

A rendszer jelenleg rendelkezik néhány alap környezettel és két tanuló algoritmussal. Viszont ezek implementálása nem a projekt része. Fontos továbbfejlesztési lehetőség lehet, újabb környezetek és tanulási algoritmusokkal kiegészíteni a jelenlegi rendszert, melyek hozzáadása a rendszer tervezésének és felépítésének köszönhetően egyszerűen eszközölhető. 

A rendszer továbbfejlesztését a webes felülettel lehetne folytatni, amely jelen formájában egy prototípus és rendelkezik a legfontosabb alapfunkcionalitásokkal, melyek szükségesek ahhoz, hogy információkat kapjunk a rendszerről és a rendszer által futtatott tesztek állapotairól, illetve az ezek által generált adatok halmazáról. Ezen funkcionalitások könnyedén kibővíthetőek a \ref{subsec:WebesFelulet} alfejezetben leírt felépítése által.

A dolgozat\cite{dolgozat} és a forráskód\cite{forras} elérhető GitHub-on. 

%%%%%%%%%%%%%%%%%%%%%%%%%%%%%%%%%%%%%%%%%%%%%%%%%%%%%%%%%%%%%%%%%%%%%%%%%%%
