%!TEX root = dolgozat.tex
%%%%%%%%%%%%%%%%%%%%%%%%%%%%%%%%%%%%%%%%%%%%%%%%%%%%%%%%%%%%
\chapter{A rendszer felépítése és használata}\label{ch:FELEPITES}

\begin{osszefoglal}
	E fejezet célja részletesen ismertetni a rendszer teljes architektúrájának felépítését és a fejlesztés során felmerülő problémák is kiemelésre kerülnek. A fejezet második részében a rendszer használatának ismertetése található. 
\end{osszefoglal}

%%%%%%%%%%%%%%%%%%%%%%%%%%%%%%%%%%%%%%%%%%%%%%%%%%%%%%%%%%%%%%%%%

\section{Nem tudom}\label{sec:Igen}

	
%%%%%%%%%%%%%%%%%%%%%%%%%%%%%%%%%%%%%%%%%%%%%%%%%%%%%%%%%%%%%%%%%%%%%%%%%%%
ENNEK MÉG NEM FOGTAM NEKI! 
\\-SZERETNÉM RÉSZLETEZNI AZ EGÉSZ ARCHITEKTÚRÁT, VALAMI DIAGRAMOKAT.
\\- LEÍRNI AZ EGÉSZ RENDSZER MŰKÖDÉSÉT, KI- KIT HÍV MEG, HOGY STB.
\\- AZ EDDIG KIMARADT KOMPONENSEKET IS MEGEMLÍTENI, MINT A PACKAGES - AMIBEN VANNAK AZ ILYEN TASKSPEC MEG ACTION IMPLEMENTÁCIÓK ÉS AZ AGENTENVIRONMENTLIST KOMPONENSRŐL IS BESZÉLNÉK AMI A LISTÁKAT TARTALMAZZA, AMELYET A WEBES FELÜLET KIÍR.
\\- ARRÓL, HOGY AZ ADATBÁZISBA MI KERÜL BE. ESETLEG, HOGY HOGYAN?!
\\- WEBES FELÜLET MEGVALÓSÍTÁS RÉSZLETEI + HASZNÁLAT
\\
\\AMENNYIBEN A TÖBBI RÉSZ RENDBEN VAN SZERETNÉM EZZEL A FEJEZETTEL KITÖLTENI A HIÁNYZÓ ~8 - 10 oldalt.
%%%%%%%%%%%%%%%%%%%%%%%%%%%%%%%%%%%%%%%%%%%%%%%%%%%%%%%%%%%%%%%%%%%%%%%%%%%





