% BEÁLLÍTÁSOK - JOBB NEM VÁLTOZTATNI
\documentclass[final]{ubb_dolgozat}
\usepackage{definitions}
\usepackage{url}
\sloppy
\frenchspacing
%%
%%


% milyen nyelveken akarunk forráskódot megjeleníteni
\lstloadlanguages{Clean,Prolog,Matlab,C,C++}

%%%%%%%%%%%%%%%%%%%%%%%%%%%%%%%%%%%%%%%%%%%%%%%
%%!!          EZT KELL VÁLTOZTATNI       !!%%%%
%%     A DOLGOZAT CÍMOLDALÁNAK ELEMEI        %%

%% MELYIK ÉVBEN ADJUK LE
\submityear{%
2015
}
%% MELYIK HÓNAPBAN ADJUK LE
\submitmonthHU{%
Július
}
\submitmonthRO{%
Iulie
}
\submitmonthEN{%
July
}

\titleHU{%
OSGI technológián alapuló tesztelési és szimulációs keretrendszer megerősítéses tanulási algoritmusok tanulmányozására
}

% Amennyiben szükséges, az alábbi sorokat ki kell komment-ezni és 
% beírni a megfelelő címeket

\titleEN{%
An OSGI-based testing and simulation framework for the study of reinforcement learning algorithms
}

\titleRO{%
Platforma de evaluare al algoritmilor de instruire prin întăriri, bazată pe OSGI
}

\author{%
Gáll Norbert
}

%%
\tutorHU{%
DR. JAKAB HUNOR,\\ EGYETEMI ADJUNKTUS\\
%{\large Babe\c{s}--Bolyai Tudományegyetem,\\
% Matematika és Informatika Kar}% ha különbözik, akkor fel kell tűntetni
}
%%
\tutorRO{%
LECTOR DR. JAKAB HUNOR\\ % {\large Universitatea Babe\c{s}--Bolyai,\\ % dacã diferã!!!
% Facultatea de Matematic\u{a} \c{s}i Informatic\u{a} }%
}
%%
\tutorEN{%
JAKAB HUNOR, PH.D,\\ ASSISTANT PROFESSOR
% {\large Babe\c{s}--Bolyai University,\\
% Faculty of Mathematics and Informatics}
}

%\includeonly{bevezet}


\begin{document}

%% ABSTRAKT
\begin{abstractEN} % ANGOL VÁLTOZAT

% a lenti részt értelemszerűen ki kell tölteni a dolgozat angol kivonatával.
% A BEGIN ... END között CSAK A SAJÁT SZÖVEG kell, hogy legyen.
% Az utolsó mondatot benne kell hagyni, mely által kijelentitek, hogy a munkátok SAJÁT.


 
	The goal of the dissertation is to create a dynamic testing and simulation environment for the evaluation of reinforcement learning algorithms on a remote server using large number of parallel running tests.
	
	The simulation environment is based on the OSGI specification, which defines a dynamic, modularized component model for building complex applications. 
Previous attempts at creating such a testing environment for an example the \texttt{RL-GLUE} project had only partial success. Although it is capable to run and evaluate various tests written in different programming languages, it is already based in obsolete technology and the project was closed a few years ago. 

	The system I made is capable of evaluating RL algorithms written in JAVA through running various experiments. All this happens with the help of a simulation environment which runs on a remote access server. The client is initiating the test which are written based on a  predefined standard, connects to the server which runs the test gets a proper feedback with the help of machine learning algorithms. 

	The server makes it possible to use a number of predefined simulation environments to perform tests that  can be run on the server independently of each other.
	 
	The framework enables the monitoring of the experiments, based on a remote method invocation API and through a web interface. 

	This work is the result of my own activity. I have neither gave nor received unauthorized assistance on this work.


\end{abstractEN}

\maketitle

% a dolgozat tartalomjegyzéke -- ez automatikusan generálódik a STRUKTÚRA alapján.
{ \baselineskip 1ex
  \parskip 1ex
  \tableofcontents
}


%%%%%%%%%%%%%%%%%%%%%%%%%%%%%%%%%%%%%%%%%%%%%%%%%%%%%%%%%%%
%%%%%%%%%%         a dolgozat tartalma         %%%%%%%%%%%%

% ajánlott külön file-okba írni az egyes fejezeteket,
% ugyanis úgy jobban át lehet látni.


% a bevezetõ fejezet FILE-ja.
%!TEX root = minta_dolgozat.tex
%%%%%%%%%%%%%%%%%%%%%%%%%%%%%%%%%%%%%%%%%%%%%%%%%%%%%%%%%%%%%%%%%%%%%%%
\chapter{Bevezető}\label{ch:BEVEZETO}
%%%%%%%%%%%%%%%%%%%%%%%%%%%%%%%%%%%%%%%%%%%%%%%%%%%%%%%%%%%%%%%%%%%%%%%
	A dolgozat témája a RoboRun projekt bemutatása, mely a megerősítéses tanulási algoritmusok futtatására és tesztelésére biztosít egy egységes szimulációs környezetet, egy távoli elérésű szerveren. 
A dolgozatban bemutatásra kerülnek a projekt főbb funkcionalitásai és a hozzá tartozó webes felület.

	A RoboRun projekt célja, egy olyan dinamikus tesztelési és szimulációs környezet felépítése, ahol megerősítéses tanulással kapcsolatos algoritmusok kipróbálására és tesztelésére van lehetőség egy távoli elérésű szerveren. E szimulációs környezet teljesen az OSGi\cite{osgi} specifikációra épül, amely egy dinamikus, modularizált komponens modellt definiál komplex alkalmazások felépítésére. E mellett teljesen egységes, hiszen a teljes rendszer saját konvenciók alapján került felépítésre. Fontos megemlíteni, hogy a rendszer könnyen elérhető és használható.
	
	A tanulás egy nagyon fontos emberi tulajdonság, mely ott van az emberek mindennapjaiban. Hiszen az ember minden nap tanul valami újat, valami új tapasztalattal gazdagodik. A gépi tanulás is az emberi tanuláson alapszik, csak más jelentéssel bír. Mondhatjuk azt, hogy egy olyan folyamat, mely során a tanuló algoritmus paraméterei és belső állapotai változnak, amelyek később meghatároznak egy döntéshozatali stratégiát. Tehát bizonyos tapasztalat alapján, melyet a belső állapotok reprezentálnak, képes a számítógép döntéseket hozni. Ennek a legfőbb nehézsége abban rejlik, hogy véges számú lépés alatt meg kell tanítani a számítógépet arra, hogy egyre jobb döntéseket hozzon végtelen sok lépés közül.
	
	A RoboRun projekt felépítése egyedinek számít a piacon, viszont létezik néhány projekt, amely rendelkezik hasonló funkcionalitásokkal. A RoboRun projekt szempontjából az Rl-Glue \footnote{\href {http://glue.rl-community.org/wiki/Main\_Page}{http://glue.rl-community.org/wiki/Main\_Page}} projektet érdemes kiemelni, hiszen ez szolgált a RoboRun projekt alapjául és számos funkcionalitása is felhasználásra került a projekt során. Az Rl-Glue\cite{rlglue} projekt szerzői Brian Tanner és Adam White. E projekt eredetileg C++ ban íródott, viszont van teljesen Java- ban megírt változata is. Az Rl-Glue egy nyelv független környezet a megerősítéses tanulási algoritmusok tanulmányozására. Kétféle protokollt kínál, az úgymond külső- illetve belső módokat. A külső mód teljesen socketeken keresztül végzi a kommunikációt, míg a belső mód az teljesen lokálisan. Ez által a belső mód sokkal gyorsabb működést eredményez. A projekt 2010- ben lezárult, viszont teljesen nyílt forráskódú, mindenki számára elérhető és használható napjainkban is. Néhány hasonló projekt: RL Toolbox\footnote{\href {http://www.igi.tu-graz.ac.at/gerhard/ril-toolbox/general/overview.html}{http://www.igi.tu-graz.ac.at/gerhard/ril-toolbox/general/overview.html}}, ClSquare\footnote{\href {http://ml.informatik.uni-freiburg.de/research/clsquare}{http://ml.informatik.uni-freiburg.de/research/clsquare}}, PIQLE\footnote{\href{http://piqle.sourceforge.net/}{http://piqle.sourceforge.net/}}. 


	A RoboRun projekt az Rl-Glue projektet tovább gondolva és funkcionalitásait felhasználva, napjaink technológiáin alapszik. Egy dinamikus és egységes környezetet biztosít, mindezt úgy, hogy a rendszer teljesen moduláris, folyamatosan és könnyen változtatható, illetve bővíthető. Mindezek mellett nagyon jól megvalósítja a komponensek egymástól való elválasztását. A projekt teljesen Java alapokon nyugszik, felhasználva az OSGi platformot. 
	
	A fő változtatásokat főként a dinamikusság, a modularitás illetve a könnyed bővíthetőség foglalja magában. Az Rl-Glue - hoz képest a teljes architektúrát újra kellett gondolni és teljesen újra felépíteni a projektet az új architektúrára, annak érdekében, hogy helyt álljon az OSGi környezetben. Az új architektúra által a projekt átláthatósága mellett, könnyen kezelhető és folyamatosan elérhető. E mellett a RoboRun projekt egy nagy előnye, hogy nagyon kis erőforrásra van a felhasználónak szüksége még komolyabb tesztek futtatásánál is, hiszen a fő logikát, tehát az erőforrás igényes részeket mind a távoli elérésű szerver futtatja. A projekt lehetőséget nyújt a felhasználók számára, egy webes felületen keresztül arra, hogy megtekinthessék az aktuálisan futó teszteket, illetve a már lefuttatott teszteredmények is megtekinthetőek.	
	
	A dolgozat hat fejezetből áll. Az első fejezet bevezetőként szolgál.
	
	 Az második fejezet röviden ismerteti a mesterséges intelligenciát, majd bemutatja a megerősítéses tanulást néhány világbeli példán keresztül. E fejezet kitér a megerősítéses tanulási kísérletek esetén felmerülő problémákra és kihívásokra. Legvégül bevezetésre kerül néhány alapfogalom.

	A harmadik fejezet a projekt által felhasznált technológiákat és eszközöket ismerteti. Bemutatja, hogy miért esett ezen technológiákra és eszközökre a választás a tervezés során, illetve a fejezet ismerteti, hogy ezek pontosan milyen szerepet játszottak a projekt megvalósítása során.	 
	
	A negyedik fejezet a RoboRun projekt alapjául szolgáló OSGi alkalmazás modellt mutatja be, ismerteti ennek architektúráját, illetve azt, hogy  miért esett az OSGi-ra a választás. E fejezet kitér arra is, hogy a projekt, hogyan használja az OSGi- t a megerősítéses tanulási kísérletek lebonyolítására.
	
	Az ötödik fejezet a rendszer teljes felépítés mutatja be. Külön kitér a webes felület felépítésének részleteire, illetve bemutatja a rendszer telepítését és használatát.
	
	A hatodik fejezet a RoboRun projekt továbbfejlesztési lehetőségeit részletezi.
\\A RoboRun projekt sikeres elkészítéséért köszönet illeti a projektvezetőt.

%!TEX root = minta_dolgozat.tex
%%%%%%%%%%%%%%%%%%%%%%%%%%%%%%%%%%%%%%%%%%%%%%%%%%%%%%%%%%%%%%%%%%%%%%%
\chapter{Alapfogalmak}\label{ch:ALAPFOGALMAK}
%%%%%%%%%%%%%%%%%%%%%%%%%%%%%%%%%%%%%%%%%%%%%%%%%%%%%%%%%%%%%%%%%%%%%%%

\begin{osszefoglal}
	E fejezet célja bemutatni röviden a megerősítéses tanulást. E mellett bemutatásra kerülnek a megerősítéses tanulási algoritmusok alaplépései, illetve a RoboRun projekttel kapcsolatos néhány alap fogalom és ezek használata a projekt során.
\end{osszefoglal}

%%%%%%%%%%%%%%%%%%%%%%%%%%%%%%%%%%%%%%%%%%%%%%%%%%%%%%%%%%%%%%%%%%%%%%%
\section{A megerősítéses tanulás}\label{sec:MEGEROSITESESTANULAS}
Már az 1950 - es évek előtt foglalkoztak mesterséges intelligencia kutatással, csak akkor még nem így nevezték. Az 1950 - es években John McCarthy\footnote{\href {http://en.wikipedia.org/wiki/Marvin_Minsky}{http://en.wikipedia.org/wiki/John\_McCarthy\_\%28computer\_scientist\%29}} megalkotja a mesterséges intelligencia kifejezést. Az intelligencia fogalmát Marvin Minsky\footnote{\href {http://en.wikipedia.org/wiki/Marvin\_Minsky}{http://en.wikipedia.org/wiki/Marvin\_Minsky}} a következőképpen definiálta: "Az intelligencia egy gyakran használt fogalom annak a rejtélynek a kifejezésére, hogy néhány önálló elem, vagy elemek felelősek a személy következtetési képességéért. Én jobban szeretem úgy elképzelni ezt, mint amely nemcsak valami különös erőt, vagy tüneményt reprezentál, hanem egyszerűen az összes mentális képességet, amelyet mi minden pillanatban megcsodálhatunk, de még nem értettünk meg." 
	
	Az ember már nagyon rég próbálkozik azzal, hogy a természettől kapott képességeit, mesterséges eszközökbe beültesse, illetve ezen eszközök segítségével kibővítse. Ennek megvalósítására a számítógépek megjelenésével nyílt igazán jó lehetőség. A számítógépek lehetőséget biztosítanak arra, hogy az emberi intelligenciát részben helyettesítsék. Rengeteg kutatás és kísérletezés folyik ennek érdekében.	
	
	A gépi tanulás is a természetből indul ki. Az élő szervezetet próbálja modellezni. Ezen algoritmusok legfőbb jellemzője az adaptációs\footnote{\href {http://hu.wikipedia.org/wiki/Adaptáció}{http://hu.wikipedia.org/wiki/Adaptáció}} tanulási képesség. A tanulás és az adaptáció valójában az élő szervezet működését jellemzi, ahogyan az ember is megszerzett ismeretek és tapasztalatok alapján cselekszik. Tanulásról beszélünk abban az esetben is, ha tapasztalatok sorozata kerül megtanulásra, illetve abban az esetben is amennyiben előző tapasztalatok alapján képes különböző döntések meghozatalára. 
	
	 Ahogyan az élő szervezeteknél, úgy a gépeknél is többfajta tanulásról beszélhetünk. Például a neurális hálók\footnote{\href {http://hu.wikipedia.org/wiki/Neurális\_hálózat}{http://hu.wikipedia.org/wiki/Neurális\_hálózat}} esetén minták alapján történik a tanulás. Ez azt jelenti, hogy nagy mennyiségű adatból próbálunk megfelelő mennyiségű ismeretet szerezni és ezáltal befolyásolni a rendszer működését. A rendszer működésének befolyásolása több dologra is irányulhat. Például, hogy a rendszer adott bemenetekre, előre megadott válaszokat produkál- e. Lehet olyan eset is, amikor azt szeretnénk tesztelni, hogy a rendszer képes- e adott bemenetekre valamilyen szabályosságot felállítani. Ezen algoritmusokat gyakran helyezik változó környezetekbe és azt tesztelik, hogy mennyire képesek alkalmazkodni az új környezethez. 	
	
	 A mesterséges intelligencia esetén beszélhetünk felügyelt, nem felügyelt és félig felügyelt tanulási módszerekről. A megerősítéses tanulás a nem felügyelt tanulási ágat képviseli. A nem felügyelt tanulás esetén nem állnak rendelkezésre adott bemenetekhez tartozó elvárt válaszok, tehát a rendszer nem rendelkezik a tanító adatok halmazával. A rendszernek a bementek és kimenetek alapján kell valamilyen viselkedést kialakítania. A környezettől nem kap semmiféle visszajelzést annak érdekében, hogy a hálózat jól vagy rosszul viselkedik-e. Ezáltal megállapítható, hogy ezen algoritmusok legfőbb jelmezője az, hogy nem előre meghatározott képességekkel rendelkeznek, amelyek csak egy adott feladat elvégzésére elegendőek, hanem képesek arra, hogy folyamatosan fejlesszék képességeiket és ezáltal képesek legyenek alkalmazkodni új és ismeretlen környezetekhez.

%%%%%%%%%%%%%%%%%%%%%%%%%%%%%%%%%%%%%%%%%%%%%%%%%%%%%%%%%%%%%%%%%%%%%%%
\section{A megerősítéses tanulás algoritmusok kipróbálásának alap lépései}\label{sec:MEGEROSITESESALOGRITMUSOK}

A megerősítéses tanulási\cite{reinfLearning} algoritmusok állapotmegfigyeléseken és jutalmakon alapulnak. Ez szintén az élő szervezetekre vezethető vissza, hiszen az állatok tapasztalatainak megszerzése is az idegrendszer állapotmegfigyelésein alapszik. Megfigyelhető például a kutyák esetében, ha egy forró tárgyhoz hozza ér hamar elkapja a mancsát, viszont még néhányszor próbálkozik. Majd néhány próbálkozás után megtanulja, hogy egy adott tárgy ha forró, ahhoz többet ne érjen hozza. 
Minden egyes algoritmus bizonyos számú epizódot hajt végre, mely rendelkezik bizonyos számú lépés sorozattal. Az algoritmusok futtatásának a célja, hogy egy optimális stratégiát alakítsanak ki a feladat megoldására illetve, hogy maximalizálják a jutalmakat. A jutalmak maximalizálása egyben a feladat elvégzésének a költségét minimalizálja.\cite{dynamicProg}

A megerősítéses tanulási algoritmusok általában egy előre definiált állapotból indulnak, az éppen aktuális problémának megfelelően. Mivel nincs semmilyen információjuk arról, hogy mi lenne a jó lépés így véletlen lépésekkel próbálkoznak. Minden egyes lépes után kiértékelik a kialakult állapotot, ezt nevezzük állapotmegfigyelésnek. Viszont a rendszernek nincs semmilyen tudomása arról, hogy a kialakult állapot jó vagy rossz. Így az algoritmusnak semmilyen alapja nem lesz arra, hogy milyen lépést kellene hozzon. Tehát az algoritmusnak szüksége van arra, hogy tudja, ha valami jó történt, illetve ha valami rossz. E miatt az egyes állapotokhoz különböző jutalmakat rendelnek. A jutalmak adása kezdetben valamilyen becslés alapján történik a jövőre nézve, ezt nevezik a jutalmak hosszútávú maximalizációjának. Egyes környezetekben a jutalmak csak a teszt végén jelennek meg, például a sakk esetén, míg más környezetben folyamatosan jönnek a jutalmak, például a pingpong esetén minden pont jutalomnak tekinthető.

A megerősítéses tanulási algoritmusok esetében három fő részt lehet elkülöníteni. Az \texttt{Agent}, azaz az Ügynök, ami valójában a tanulási algoritmus. Az \texttt{Environment}, azaz a Környezet, amely meghatározza az adott tesztet, például a sakk problémája. Az \texttt{Experiment}, azaz a Kísérlet, amely meghatározza az epizódok számát, illetve az epizódokon belül a lépések számát. E három fő részről és az ezek közötti kommunikáció részletesebb leírását az \ref{sec:AlapfogalmakBevezetese} szekció tartalmazza.

%%%%%%%%%%%%%%%%%%%%%%%%%%%%%%%%%%%%%%%%%%%%%%%%%%%%%%%%%%%%%%%%%%%%%%%
\subsection{Kihívások a kísérletek futtatása során}
A megerősítéses tanulási algoritmusok futtatása saját számítógépen nem a legjobb megoldás. Ezen kísérletek futtatása egy időigényes folyamat lehet, például egy robot esetén, melynek teljes mozgását reprezentálni szeretnénk. A robot összes mozgásának reprezentálása nagyon sok lépést vehet igénybe, annak érdekében, hogy sikeresen kialakításra kerüljön az optimális mozgási stratégia. A kísérletek futtatása során felmerül az a probléma, hogy egy adott kísérlet több példányát  kellene végrehajtani egyidejűleg, illetve több különböző kísérlet példányait egyszerre mindezt online. Amennyiben egy kísérlet futási ideje több óra, nap esetleg hét is lehet, a fejlesztők nem várhatnak egy adott kísérlet befejezésére, hiszen akkor az előre haladás nagyon nagyon lassú lenne. Az időigényesség mellett fontos megemlíteni, hogy komplexebb szimulációs környezetek esetén, melyet egy komplex tanulási algoritmus irányít, nagyon nagy erőforrás igényre lehet szükség. Amennyiben már egy algoritmus végrehajtása nagy erőforrás igényű lehet, akkor több algoritmus egyidejű végrehajtása esetén óriási erőforrásigényekről beszélhetünk. A kísérletek nagy mennyiségű adatokat generálnak, melyekre a későbbi kiértékelés és esetleges összehasonlítások során szükség lehet. Ezen adatok tárolására, adatbázisokra lehet szükség a könnyed hozzáférés érdekében. Ezen kihívások azt eredményezik, hogy a számítógép állandóan be kell legyen kapcsolva, a nagy erőforrás igény miatt másra nem is lehetne használni, illetve nehezen managelhetőek lennének a tesztek.  

Ennek megoldására egy olyan szimulációs környezet megvalósítása kínál lehetőséget, amely egy távoli, jól felszerelt szervergépen fut, mely folyamatosan elérhető az interneten keresztül és rendelkezik a megfelelő erőforrásokkal. E környezet dinamikusan kell működjön, hiszen egyszerre több teszt futtatását kell lehetővé tegye. E mellett rendelkeznie kell valamilyen adatbázis kapcsolattal, ahova minden teszt esetén elmenti az adatokat. Ezen tulajdonságok mellett, elengedhetetlen a szimulációs környezet számára a könnyed bővíthetőség és módosíthatóság tulajdonsága, amely által a már meglévő kísérletek könnyedén megváltoztathatóak az új elképzelések alapján, illetve az új kísérletek hozzáadása könnyedén eszközölhető.

%%%%%%%%%%%%%%%%%%%%%%%%%%%%%%%%%%%%%%%%%%%%%%%%%%%%%%%%%%%%%%%%%%%%%%%
\section{Alapfogalmak bevezetése}\label{sec:AlapfogalmakBevezetese}

A projekt megvalósítása során a szerző megismerte az  Rl-Glue projekt elveit és annak funkcionalitásait, melyek meghatározó szerepet töltenek be a RoboRun projektben is. A három alap komponens mindkét projekt esetén  az \texttt{Agent}(Ügynök), \texttt{Environment}(Környezet) és az \texttt{Experiment}(Kísérlet). Ezen komponensek egymással való interakciója révén nyílik lehetőségünk futtatni illetve tesztelni a megerősítéses tanulási algoritmusokat. 

	Az \texttt{Agent} komponens valójában a tanulási algoritmus, amely kiszabja a feladatokat és az ezekre vonatkozó megszorításokat egy adott iterációra vonatkozóan. Az \texttt{Agent} jutalmat(reward) kap minden egyes iteráció után arra vonatkozóan, hogy a probléma megoldásának szempontjából mennyire volt hatékony a kiszabott feladat, illetve az erre vonatkozó megszorítás. Mivel nem tudhatja az algoritmus, hogy melyik a helyes módszer a probléma megoldására, ezért találgatnia kell. Időnként új cselekvéseket is kell próbálnia, majd az ezekből megszerzett tudást, ami esetünkben a jutalom, optimális módon felhasználnia a következő cselekvés meghatározására. Például \texttt{RandomAgent}, mely valójában nem is tanulási algoritmus, hiszen véletlenszerű lépések alapján próbál az egyes környezetekben megoldást találni a problémára. Viszont gyakran használják egyes környezetek kipróbálására.
	
	Az \texttt{Environment} komponens feladata végrehajtani az \texttt{Agent} komponens által meghatározott feladatokat és az ezekre vonatkozó megszorításokat az adott problémára. A végrehajtás során következtetéseket(observation) von le minden egyes állapotról. Majd ezen következtetések alapján jutalmakat(reward) határoz meg. Mivel bizonytalan a környezet,    valami becslést kell alkalmaznia a jövőre nézve, így kezdetben lehet, hogy egy jó lépésért nem kapjuk meg a megfelelő jutalmat. Viszont minél jobban megismerjük a környezetet, annál pontosabb lesz egy lépésért vagy lépés sorozatért járó jutalom. A jutalom egy számban fejezhető ki, amely egy adott intervallumban mozog. Ha az intervallum felső határához közelít a szám akkor pozitív visszajelzést kaptunk az adott lépés vagy lépes sorozat után, amennyiben az intervallum alsó határához közelít a szám, negatív a visszajelzés. Például a \texttt{MountainCar} környezet, amelyben az a feladata a tanuló algoritmusnak, hogy egy bizonyos kezdeti pozícióból eljusson egy kijelölt végső pozícióba a lehető leggyorsabban. Ehhez meg kell tanulnia a szakadékban előre és hátra mennie egészen addig, amíg olyan sebességre tesz szert, hogy elegendő a végpontba való eljutáshoz.
	
	Az \texttt{Experiment} komponens irányítja a teljes kísérlet végrehajtását. E komponens nincs direkt kapcsolatban az \texttt{Agent} és az \texttt{Environment} komponensekkel. Van köztük egy köztes réteg, amely végzi a kommunikációt e három komponens között.  Az \texttt{Experiment} komponensben van meghatározva a lépések száma egy adott iterációban, illetve az iterációk száma is. Fontos azon szerepe is az \texttt{Experiment} komponensnek, hogy a végső eredményeket ő kapja meg az \texttt{Agent} illetve az \texttt{Environment} komponensektől a köztes rétegen keresztül.
A fent említett köztes réteg az Rl-Glue projekt esetén az úgynevezett RL-Glue mely a teljes kommunikáció lebonyolítását végzi a komponensek között illetve létrehozza a hálózati kommunikációhoz szükséges objektumokat. Az Rl-Glue projekt alap kommunikációját az \texttt{Agent}, \texttt{Environment}, \texttt{Experiment} és \texttt{RL-Glue} között a \ref{fig:RlGlueKommunikacio} ábra szemlélteti.

\begin{figure}[h!]
  \centering
  \pgfimage[width=0.7\linewidth]{images/glueConnection}
  \caption[Példa képek beszúrására]%
  {Rl-Glue projekt komponensek közti kommunikációja:\\
  {\white .}\hfill\url{http://rl-glue.googlecode.com/svn/trunk/docs/html/index.html}}
  \label{fig:RlGlueKommunikacio}
\end{figure}

A RoboRun projekt esetén az RL-Glue komponens helyét felváltja a \texttt{RoboCommunication} komponens, a funkcionalitásokat tekintve az alap ötlet megmaradt viszont számos új dologgal ki lett bővítve, illetve új funkcionalitások kerülnek használatra. Számos funkcionalitás található az RL-Glue projektben, amelyre a RoboRun projektben nincs szükség, így ezen funkcionalitások teljesen ki lettek hagyva. Ilyen például a hálózat alapú kommunikáció.

	Az \texttt{Agent}, \texttt{Environment}, \texttt{Experiment} és a \texttt{RoboCommunication}  komponensek interakciójának sorrendje hasonlóan működik, mint az Rl - Glue projektben megvalósított elgondolás, viszont a RoboRun projekt esetén minden egyes komponens  egy szolgáltatás az OSGi konténerben, amelyek képesek egymással kapcsolatba lépni az OSGi szolgáltatásokon keresztül, így megvalósítva a dinamikus, komponens alapú modell kivitelezését és a több teszt egy időben való futtatási lehetőségét. E mellett ezen szolgáltatások elérése korlátozott, a többi szolgáltatás számára, amely biztonsági okokból is előnyös a rendszerre nézve. Az OSGi konténer futtatásáért egy GlassFish\citep{glassfish} szerver a felelős. Az OSGi specifikáció részletes ismertetése a \ref{ch:OSGI} fejezetben olvasható.
	
	 A konténerben egyszerre több előre definiált \texttt{Agent} és \texttt{Environment} lehet telepítve, ezek száma nincs korlátozva, annyi telepíthető belőlük amennyi még nem okoz gondot a szervert futtató számítógép hardver konfigurációjának. Bármikor módosítható, törölhető vagy teljesen új \texttt{Agent} és \texttt{Environment} is hozzáadható a konténerhez anélkül, hogy a szervert meg kellene állítani vagy újra kellene indítani. A \texttt{RoboCommunication} komponensből egyet tartalmaz a rendszer, hiszen ez az egy szolgáltatás képes kielégíteni számtalan \texttt{Agent} és \texttt{Environment} komponens példányt egy időben, mindezt a projekt architektúrájának és az OSGi keretrendszerben rejlő lehetőségek által. Az Experimentekből is egyszerre több lehet telepítve, hiszen ezek felelnek egy egy teszt indításáért. Egyszerre több tesztet is lehet indítani, vagy lehetőség van arra is, hogy különböző időközönként telepítsünk egy- egy \texttt{Experiment} komponenst. A RoboRun projekt alap komponensei közötti kommunikációt szemlélteti a \ref{fig:OsgiAlap} ábra.

\begin{figure}[h!]
  \centering
  \pgfimage[width=1\linewidth]{images/alapKomponens}
  \caption[RoboRun alap komponensek]%
  {A RoboRun projekt alap komponensek közti kommunikáció:\\
  {\white .}\hfill\url{}}
  \label{fig:OsgiAlap}
\end{figure}


Egy kísérlet futtatásához szükségünk van egy \texttt{Agent}, egy \texttt{Environment}, egy \texttt{Experiment} és egy \texttt{RoboCommunication} komponensre. A \texttt{RoboCommunication} komponens esetén még szükség van néhány függőségre az adatbázishoz való hozzáférés biztosítására, illetve a webes felület adatainak féltöltésé érdekében, ezekről részletesebben \ref{ch:Felepites} fejezetben lehet megismerkedni. A kísérlet belépési pontjaként az Experiment komponens szolgál, hiszen ő határozza meg, hogy melyik \texttt{Agent}, illetve \texttt{Environment} példánnyal szeretne dolgozni, illetve az \texttt{Experiment} határozza meg a szükséges lépések számát iterációnként és az iterációk számát. Az \texttt{Experiment} komponens lekérdezi a szükséges szolgáltatásokat és átadja a RoboCommunication szolgáltatásnak a lekérdezett \texttt{Agent} és \texttt{Environment} szolgáltatás példányokat. A \texttt{RoboCommunication} komponens fogja biztosítani e három komponens között a folyamatos kommunikációt a teszt futtatása során. A tesztek állapotának nyomon követésére lehetőség van a webes felületen keresztül, illetve a tesztek befejeztével megtekinthetőek a futása során létrejött adatok.

	A projekt architektúrája követi a OSGi szabványokat, így minden szolgáltatásnak implementálnia kell egy interfészt, mely egy külön batyuban található. Az interfész közzéteszi a szolgáltatások számára a csomagját. A csomag importálása által a szolgáltatások megvalósíthatják az adott interfészt. Minden egyes \texttt{Agent} kötelező módon implementálja az \texttt{AgentInterface} -t és minden \texttt{Environment} kötelező módon implementálja az \texttt{EnvironmentInterface} -t. Ez által valósul meg a szimulációs környezet egységessége, amely egy egyedi konvencióra épül. Nem létezhet olyan \texttt{Agent} vagy \texttt{Environment} példány a rendszerben, amely nem implementálja a neki megfelelő interfészt. Amennyiben mégis telepítésre kerül egy ilyen szolgáltatás, nem fog működni, mert egy szolgáltatás lekérdezése csak az általa implementált interfész által lehetséges. Az interfészek és a hozzájuk tartozó implementációkat a \ref{fig:alapUML} ábra szemléltet.

\begin{figure}[h]
  \centering
  \pgfimage[width=1\linewidth]{images/alapUML}
  \caption[Példa képek beszúrására]%
  {Agent és Environment Interfész és implementáció\\
  {\white .}\hfill\url{}}
  \label{fig:alapUML}
\end{figure}


%%%%%%%%%%%%%%%%%%%%%%%%%%%%%%%%%%%%%%%%%%%%%%%%%%%%%%%%%%%%%%%%%%%%%%%

% %!TEX root = dolgozat.tex
%%%%%%%%%%%%%%%%%%%%%%%%%%%%%%%%%%%%%%%%%%%%%%%%%%%%%%%%%%%%%%%%%%%%%%%
\chapter{Felhasznált módszerek és eszközök}\label{ch:MODSZEREK_ES_ESZKOZOK}

\begin{osszefoglal}
	Minden nagyobb projekt esetén a fejlesztőknek szükségük van arra, hogy a megfelelő felkészültségük és találékonyságuk mellett, figyelmet fordítsanak a hatékony munkára is. Ehhez szükségük lehet különböző verziókövető rendszerek használatára, projektmenedzsment eszközökre és build rendszerekre.
\end{osszefoglal}


%%%%%%%%%%%%%%%%%%%%%%%%%%%%%%%%%%%%%%%%%%%%%%%%%%%%%%%%%%%%%%%%%%%%%%%%%%%
\section{Verziókövetés}

Napjainkban egyre nagyobb szükség van arra, hogy egy projekt esetén a munka könnyedén megosztható és hordozható legyen a fejlesztők közt. E mellett nagyon fontos a fejlesztési folyamat monitorizálása. Ezen technológiák nélkül szinte elképzelhetetlen a szoftverfejlesztés, úgy csoportos környezetben mint egyedül.

	E célra fejlesztették  ki a verziókövető rendszereket és a projektmenedzsment eszközöket melyek által könnyedén megoszthatóvá válik a fejlesztői munka és folyamatosan ellenőrizhető a fejlesztés folyamata.  
	
	A verziókövető rendszer által  folyamatosan nyomon követhető a projekt fejlődése és ellenőrizhető az egyénenkénti haladás is. Könnyed visszaállítási lehetőséget biztosít arra az esetre, ha valami történne a lokális gépünkön tárolt forrás állományokkal vagy ha bármi hiba történne a fejlesztés során ami visszaállítást igényel. Legnagyobb haszna a verzió követő rendszereknek, az olyan projekteknél van, amelyet több fejlesztő fejleszt egyszerre. Hiszen általában ilyenkor a projekt teljes forrásállománya egy központi tárolóban van elhelyezve ahová mindenki beteszi a változtatásait. Így nagyon egyeszűen követhető, hogy melyik fejlesztő milyen fázisban tart. 
	
	A RoboRun projekt fejlesztése során a forrásállományok tárolására és a fejlesztés nyomkövetésére felhasznált verziókövető rendszer a Git\citep{git}, amely nyílt forráskódú és teljesen ingyenes. Könnyedén megoszthatóak a forrásállományok. A projekt szerzője által használt kliensalkalmazás a  TortoiseGit\citep{tortoisegit}. A TortoiseGit szintén ingyenes szoftver, melyet szükséges telepíteni. Használata egyszerű. A konzol mellett, rendelkezik egy felhasználóbarát grafikus felülettel is, mely még inkább megkönnyíti a használatát.

%%%%%%%%%%%%%%%%%%%%%%%%%%%%%%%%%%%%%%%%%%%%%%%%%%%%%%%%%%%%%%%%%%%%%%%%%%%
\section{Projektmenedzsment}

"A projektmenedzsment az erőforrások szervezésével és azok irányításával foglalkozó szakterület, melynek célja, hogy az erőforrások által végzett munka eredményeként egy adott idő- és költségkereten belül sikeresen teljesüljenek a projekt céljai." (forrás: http://hu.wikipedia.org/wiki/Projektmenedzsment)

	A projektmenedzsment eszközök fő célja tehát, hogy a fejlesztők a specifikáció által meghatározott feladatot adott idő - és költségkereten belül sikeresen tudják teljesíteni. Ennek érdekében számos hasznos funkcionalitást biztosítanak. Ilyen funkcionalitások például, különböző feladatkörök kiosztása, különböző feladatok kiosztása, egy adott folyamatra szánt idő meghatározása, a fejlesztő által eltöltött munkaidő egy adott rész megvalósításával. Mindezek mellett kommunikációs lehetőséget biztosít a fejlesztők között. Lehetőség ad feltölteni dokumentumokat, diagramokat, segédanyagokat a projekthez, ez által megkönnyítve a fejlesztők munkáját. 
	
	A RoboRun projekt fejlesztése során alkalmazott projektmenedzsment eszközként a Redmine\citep{redmine} webes menedzsment eszköz szolgált. A Redmine egy teljesen  nyílt forráskódú és platform független projektmenedzsment rendszer. Egyszerű és letisztult felülete révén könnyen kezelhető. Rengeteg funkcionalítást nyújt, mely nagy segítség lehet a különböző projektek fejlesztése során. A  Redmine által nyújtott néhány fontosabb funkcionalitás: naptár, e-mail értesítés, szerepkör szerinti hozzáférés, wiki és fórum, pluginok engedélyezés, adatbázisok támogatása, stb.



%%%%%%%%%%%%%%%%%%%%%%%%%%%%%%%%%%%%%%%%%%%%%%%%%%%%%%%%%%%%%%%%%%%%%%%%%%%
\section{Build rendszer}

A build rendszereket többnyire projektek menedzselésére és a build folyamat automatizálására alkalmazzák. 

	A RoboRun projekt fejlesztése során alkalmazott build rendszer a Maven\cite{maven}, amelyet Jason van Zyl készített 2002-ben. A Maven egy nyílt forráskódú, platform független eszköz. Leggyakoribb felhasználása a Java nyelvben írt projektek esetében történik. A Maven konfigurációs modellje XML alapú, e mellett bevezetésre került a POM(Project Object Model). A POM az adott projekt szerkezeti vázának teljes leírását tartalmazza és a modulokat azonosítókkal látja el. Tehét a POM egy projekt leírását tartalmazza és a projekthez tartozó összes függőség listáját. Ezen függőségeket a Maven a saját központi tárolójából tölti le a projekt buildelése során. A POM esetén a lépéseket céloknak nevezik. A célok lehetnek előre definiáltak, mint például a forráskód csomagolása és fordítása vagy lehetnek a felhasználó által meghatározott célok.  Mindezt a pom.xml állomány által valósul meg, amely tartalmazza ezeket az információkat. 
	
	A RoboRun projekt esetén a Maven build eszköz legfontosabb szerepe a függőségek, célok és pluginok kielégítése a build folyamat során, hiszen a Maven saját függőség kezelő rendszerrel rendelkezik, amely  a build - elés során letölti a központi tárolóból az előre megadott függőségeket és elhelyezi a lokális tárolóban, ahonnan a jövőben használni fogja. E mellett a Maven lehetőséget nyújt a projekt moduljainak azonosítására a groupID, az artifactID és a verzió szám révén. A groupID logikai csoportokba szervezi a komponenseket, az artifactID minden komponenst egyedi azonosítóval lát el és a verzió az éppen aktuális verziószámot takarja a komponensek esetén.

%%%%%%%%%%%%%%%%%%%%%%%%%%%%%%%%%%%%%%%%%%%%%%%%%%%%%%%%%%%%%%%%%%%%%%%%%%%
\chapter{Felhasznált technológiák és eszközök}\label{ch:TechnologiakEsEszkozok}

\begin{osszefoglal}
	A fejezet ismerteti a RoboRun projekt által felhasznált alkalmazás szervert és ennek szerepét, illetve a webalkalmazáshoz felhasznált technológiát, mely a projekt webes felületének megvalósításában játszott szerepet. A fejezet célja, hogy bemutassa azon eszközöket, amelyek meghatározó szerepet töltöttek be a projekt megalkotása során.
\end{osszefoglal}


\section{Alkalmazásszerver}\label{sec:Glassfish}
	
	A RoboRun projekt megvalósításához elengedhetetlen egy alkalmazásszerver használata, hiszen az OSGi konténernek szüksége van egy alkalmazásszerverre, amely képes futtatni és ez által folyamatosan elérhetővé tenni az alkalmazást az interneten keresztül. Bővebben az OSGi specifikációról a \ref{ch:OSGI}. fejezetben lehet olvasni.
	
	 Az alkalmazásszerver nem más mint egy szoftver keretrendszer, mely lehetőséget biztosít tetszőleges alkalmazások futtatására. A Java Enterprise Edition specifikáció több referencia implementációval rendelkezik. Ilyen referencia implementációk például: JBoss As\footnote{\href {http://jbossas.jboss.org/}{http://jbossas.jboss.org/}}, GlassFish\footnote{\href {https://glassfish.java.net/}{https://glassfish.java.net/}}, WebSphere\footnote{\href {http://www.ibm.com/software/websphere}{http://www.ibm.com/software/websphere}}, WebLogic\footnote{\href {http://www.oracle.com/technetwork/middleware/weblogic/overview/index-085209.html}{http://www.oracle.com/technetwork/middleware/weblogic/overview/index-085209.html}}.
	
	A RoboRun projekt tervezésekor ezen alkalmazásszerverek közül a Glassfish referencia implementációja tűnt megfelelőnek, mely az Oracle tulajdonába tartozik és teljesen nyílt forráskódú, viszont vásárolható hozzá kereskedelmi licensz támogatás, amelyben az Oracle saját megoldásai is helyet kapnak, például képes teljes domainek mentésére és visszaállítására. A Glassfish fontos tulajdonsága, hogy képes OSGi konténerek kezelésére. A RoboRun projekt esetén a Glassfish alkalmazásszerverre telepített OSGi konténerben vannak elhelyezve a különálló komponensek, amelyek együtt alkotják a RoboRun projektet. A Glassfish alkalmazásszerver révén távolról is telepíthetőek könnyedén új komponensek, illetve használhatóak. A Glassfish manageli a RoboRun projekt webes felületét is.

\section{Webalkalmazás}\label{sec:Vaadin}

	A projekt webes felületének megvalósítására rengeteg módszer, eszköz és keretrendszer áll rendelkezésre, melyek közül lehet választani. A RoboRun projekt tervezésekor fontos szempont volt a könnyed és gyors használhatóság, illetve a dinamikus módosíthatóság tulajdonsága. Ezek alapján a Vaadin\cite{vaadin} webalkalmazás-keretrendszerre esett a választás.
	
	A Vaadin egy olyan Java webalkalmazás-keretrendszer, amely lehetőséget biztosít gazdag webalkalmazások\footnote{\href {http://hu.wikipedia.org/wiki/Rich\_Internet\_Application}{http://hu.wikipedia.org/wiki/Rich\_Internet\_Application}} fejlesztésére. A Vaadin lehetőséget nyújt a felület Java nyelvben történő implementálására, illetve egy AJAX alapú kommunikációs modellt biztosít. A Vaadin architektúra két fő részből tevődik össze: a kliens oldali részből, amely tartalmazza a Google Web Toolkit -et\footnote{\href {http://hu.wikipedia.org/wiki/Google\_Web\_Toolkit}{http://hu.wikipedia.org/wiki/Google\_Web\_Toolkit}} és amely a Java kódot, JavaScript kódra fordítja, illetve a szerver oldali részből, amely JavaServlet technológiát használ.
A szerver oldali rész tartalmazza a felhasználó felület létrehozásához szükséges komponenseket. Ezen komponensek nagyon hasonlítanak a Java-ban írt standard alkalmazásoknál használt AWT, illetve SWING komponensekre. A Vaadin-os komponensek is figyelőket és eseményeket használnak. Az alapértelmezett komponens és téma készlet mellett, telepíthetőek különböző kiegészítők a még könnyebb használat és a még látványosabb felhasználói élmény érdekében. A komponensek személyre szabhatóak a CSS, HTML5, JavaScript technológiák felhasználása által.

\subsection{A Vaadin architektúra}\label{sec:Vaadin architektura}

A Vaadin webalkalmazás-keretrendszer architektúrája két fő részre osztható fel. A kliens oldal által valósul meg a megjelenítés, amely JavaScript formájában jelenik meg a böngészőben. A szerver oldali rész  mely biztosítja a komponensek könnyed elérését és használatát.
\begin{figure}[h!]
  \centering
  \pgfimage[width=0.8\linewidth]{images/vaadinArchitecture}
  \caption[Vaadin keretrendszer architektúrája]%
  {Vaadin architektúra\\
  {\white .}\hfill\url{https://vaadin.com/book/-/page/architecture.html}}
  \label{fig:vaadinArchitektura}
\end{figure} 

A \ref{fig:vaadinArchitektura}~ ábra szemlélteti a Vaadin keretrendszer architektúráját.
A kliens oldal áll az architektúra legfelső részén, hiszen a kliens közvetlen ezzel kerül kapcsolatba. Ez a réteg a belépési pont, ahonnan adatok érkeznek, melyeket a szerver oldali rész fele kell közvetíteni. A kliens oldali részben megtalálható a kliens oldali felhasználói felület és az ehhez tartozó widgetek listája, amelyek a megjelenítésért felelősek. 
A szerver felelős az üzleti logika megvalósításáért. A Service réteg valósítja meg a kommunikációt a Back-end réteg és a kliens réteg között. 


\subsection{A Vaadin komponensek}\label{sec:Vaadin komponensek}

A Vaadin webalkalmazás-keretrendszer egy előre definiált komponens gyűjteményt biztosít a fejlesztők számára, a könnyebb munka érdekében. Ezen komponensek használata egyszerű, hiszen nagyon hasonlítanak az AWT és a SWING  - es komponensekhez. Az alap gyűjteményben található komponensek által felépíthető egy web alkalmazás teljes felhasználói felülete. Ezen komponensek bővíthetőek, teljesen személyre szabhatóak a CSS, HTML5, JavaScript technológiák által. A Vaadin keretrendszer lehetőséget biztosít teljesen új komponensek definiálására is.
A Vaadin keretrendszer alap komponensei közötti kapcsolatot, illetve összefüggéseket a \ref{fig:vaadinComponents}~ábra szemlélteti. 
\begin{figure}[h!]
  \centering
  \pgfimage[width=1\linewidth]{images/vaadinComponents}
  \caption[Vaadin keretrendszer komponensei]%
  {Vaadin komponensek\\
  {\white .}\hfill\url{https://inftec.atlassian.net/wiki/display/TEC/Vaadin}}
  \label{fig:vaadinComponents}
\end{figure} 

A legfelső réteg a \texttt{Component} interfész, melyet az \texttt{AbstractComponent} absztrakt osztály implementál. Az \texttt{AbstractComponent} közös tulajdonságokkal látja el az őt származtató komponenseket. Az \texttt{AbstractComponent} osztályból származtatva van néhány egyszerű komponens, például a \texttt{Label}(Címke). A \texttt{Component} interfész mellet, megtalálható a  \texttt{Field} interfész is, amely örökli a \texttt{Component} interfészt. Az \texttt{AbstractField} absztrakt osztály implementálja a  \texttt{Field} interfész és örökli az \texttt{AbstractComponent} osztály tulajdonságait. Az \texttt{AbstractField} osztály a kijelöléssel, navigálással kapcsolatos komponensek alapjait képezi, például \texttt{Button}(Gomb). Az \texttt{AbstractComponent} osztály tulajdonságait örökli, az \texttt{AbstractComponentContainer} osztály is. Az \texttt{AbstractComponentConainer} osztályból származnak a konténer - alapú komponensek, például a  \texttt{Panel}, illetve az elrendezést elősegítő komponensek, például \texttt{VerticalLayout}(Függőleges elrendezés).



%%%%%%%%%%%%%%%%%%%%%%%%%%%%%%%%%%%%%%%%%%%%%%%%%%%%%%%%%%%%%%%%%%%%%%%%%%%
\section{Verziókövetés}

Napjainkban egyre nagyobb szükség van arra, hogy egy projekt esetén a munka könnyedén megosztható és hordozható legyen a fejlesztők közt. E mellett nagyon fontos a fejlesztési folyamat monitorizálása. Ezen technológiák nélkül szinte elképzelhetetlen a szoftverfejlesztés, úgy csoportos környezetben mint, egyénileg.

	E célra fejlesztették  ki a verziókövető rendszereket és a projektmenedzsment eszközöket, melyek által könnyedén megoszthatóvá válik a fejlesztői munka és folyamatosan ellenőrizhető a fejlesztés folyamata.  
	
	A verziókövető rendszer által  folyamatosan nyomon követhető a projekt fejlődése és ellenőrizhető az egyénenkénti haladás is. Könnyed visszaállítási lehetőséget biztosít arra az esetre, ha valami történne a lokális gépünkön tárolt forrás állományokkal vagy ha bármi hiba történne a fejlesztés során ami visszaállítást igényel. Legnagyobb haszna a verzió követő rendszereknek, az olyan projekteknél van, amelyet több fejlesztő fejleszt egyszerre. Hiszen általában ilyenkor a projekt teljes forrásállománya egy központi tárolóban van elhelyezve, ahová mindenki beteszi a változtatásait. Így nagyon egyszerűen követhető, hogy melyik fejlesztő milyen fázisban tart. 
	
	A RoboRun projekt fejlesztése során a forrásállományok tárolására és a fejlesztés nyomkövetésére felhasznált verziókövető rendszer a Git\citep{git}, amely nyílt forráskódú és teljesen ingyenes. Könnyedén megoszthatóak a forrásállományok. A projekt szerzője által használt kliensalkalmazás a  TortoiseGit\citep{tortoisegit}. A TortoiseGit szintén ingyenes szoftver, melyet szükséges telepíteni. Használata egyszerű. A konzol mellett, rendelkezik egy felhasználóbarát grafikus felülettel is, mely még inkább megkönnyíti a használatát.

%%%%%%%%%%%%%%%%%%%%%%%%%%%%%%%%%%%%%%%%%%%%%%%%%%%%%%%%%%%%%%%%%%%%%%%%%%%
\section{Projektmenedzsment}

"A projektmenedzsment az erőforrások szervezésével és azok irányításával foglalkozó szakterület, melynek célja, hogy az erőforrások által végzett munka eredményeként egy adott idő- és költségkereten belül sikeresen teljesüljenek a projekt céljai."\cite{projektmenedzsment}

	A projektmenedzsment eszközök fő célja tehát, hogy a fejlesztők a specifikáció által meghatározott feladatot adott idő - és költségkereten belül sikeresen tudják teljesíteni. Ennek érdekében számos hasznos funkcionalitást biztosítanak. Ilyen funkcionalitások például, különböző feladatkörök kiosztása, különböző feladatok kiosztása, egy adott folyamatra szánt idő meghatározása, a fejlesztő által eltöltött munkaidő egy adott rész megvalósításával. Mindezek mellett kommunikációs lehetőséget biztosít a fejlesztők között. Lehetőség ad feltölteni dokumentumokat, diagramokat, segédanyagokat a projekthez, ez által megkönnyítve a fejlesztők munkáját. 
	
	A RoboRun projekt fejlesztése során alkalmazott projektmenedzsment eszközként a Redmine\citep{redmine} webes menedzsment eszköz szolgált. A Redmine egy teljesen  nyílt forráskódú és platform független projektmenedzsment rendszer. Egyszerű és letisztult felülete révén könnyen kezelhető. Rengeteg funkcionalítást nyújt, mely nagy segítség lehet a különböző projektek fejlesztése során. A  Redmine által nyújtott néhány fontosabb funkcionalitás: naptár, e-mail értesítés, szerepkör szerinti hozzáférés, wiki és fórum, pluginok engedélyezése, adatbázisok támogatása, stb.



%%%%%%%%%%%%%%%%%%%%%%%%%%%%%%%%%%%%%%%%%%%%%%%%%%%%%%%%%%%%%%%%%%%%%%%%%%%
\section{Build rendszer}

A build rendszereket többnyire projektek menedzselésére és a build folyamat automatizálására alkalmazzák. 

	A RoboRun projekt fejlesztése során alkalmazott build rendszer a Maven\cite{maven}, amelyet Jason van Zyl készített 2002-ben. A Maven egy nyílt forráskódú, platform független eszköz. Leggyakoribb felhasználása a Java nyelvben írt projektek esetében történik. A Maven konfigurációs modellje XML alapú, e mellett bevezetésre került a POM(Project Object Model). A POM az adott projekt szerkezeti vázának teljes leírását tartalmazza és a modulokat azonosítókkal látja el. Tehát a POM egy projekt leírását tartalmazza és a projekthez tartozó összes függőség listáját. Ezen függőségeket a Maven a saját központi tárolójából tölti le a projekt buildelése során. A POM esetén a lépéseket céloknak nevezik. A célok lehetnek előre definiáltak, mint például a forráskód csomagolása és fordítása vagy lehetnek a felhasználó által meghatározott célok.  Mindezt a pom.xml állomány által valósul meg, amely tartalmazza ezeket az információkat. 
	
	A RoboRun projekt esetén a Maven build eszköz legfontosabb szerepe a függőségek, célok és pluginok kielégítése a build folyamat során, hiszen a Maven saját függőség kezelő rendszerrel rendelkezik, amely  a build - elés során letölti a központi tárolóból az előre megadott függőségeket és elhelyezi a lokális tárolóban, ahonnan a jövőben használni fogja. E mellett a Maven lehetőséget nyújt a projekt moduljainak azonosítására a groupID, az artifactID és a verzió szám révén. A groupID logikai csoportokba szervezi a komponenseket, az artifactID minden komponenst egyedi azonosítóval lát el és a verzió az éppen aktuális verziószámot takarja a komponensek esetén.

%%%%%%%%%%%%%%%%%%%%%%%%%%%%%%%%%%%%%%%%%%%%%%%%%%%%%%%%%%%%%%%%%%%%%%%%%%%


\section{További felmerülő problémák megoldása}\label{sec:TovábbiProblemak}

	A RoboRun projekt esetén szükséges a tesztek futtatásakor az egyes hibaüzenetek megtekintésének lehetősége, melyet a naplózás biztosít. Naplózáshoz az SLF4J\footnote{\href {http://www.slf4j.org/}{http://www.slf4j.org/}} - LOG4J\footnote{\href {http://logging.apache.org/log4j/2.x/}{http://logging.apache.org/log4j/2.x/}} párosítás került használatra. Az SLF4J több naplózási keretrendszer fölött képez absztrakciós szintet, így több különböző naplózási implementációt vehetünk igénybe általa. A LOG4J implementáció az Apache licenc áll. Az SLF4J naplózási keretrendszernek a legfőbb előnye, hogy a LOG4J bármikor könnyen lecserélhető, bármilyen más implementációra. 
	A projekt esetén adatbázis menedzsment rendszerként a MySQL\footnote{\href {https://www.mysql.com/}{https://www.mysql.com/}} szolgált, amely többfelhasználós és többszálú SQL - alapú adatbázis-kezelő rendszer. 
	Egy másik felmerülő probléma, az RMI(Remote Method Incovation)\footnote{\href {http://en.wikipedia.org/wiki/Java\_remote\_method\_invocation}{http://en.wikipedia.org/wiki/Java\_remote\_method\_invocation}} problémája. Az eredeti elképzelés szerint a kliens RMI - n keresztül csatlakozott volna a szerverhez, elkérve onnan a telepített szolgáltatások listáját. A listából kiválasztva a neki megfelelő szolgáltatásokat, indított volna teszteket. Tehát az alkalmazás \texttt{Experiment} része teljesen az OSGi konténeren kívül lett volna elérhető. Ennek megvalósítása nem lehetséges, a miatt, hogy az RMI kapcsolaton keresztül elküldött objektumok szerializálásra kerülnek. Az OSGi szolgáltatásokat pedig nem lehet szerializálni, tehát nem elérhetőek a konténeren kívül. Erről részletesebb leírás a \ref{ch:KOVETKEZTETESEK}. fejezetben található.
	
	
%!TEX root = dolgozat.tex
%%%%%%%%%%%%%%%%%%%%%%%%%%%%%%%%%%%%%%%%%%%%%%%%%%%%%%%%%%%%%%%%%%%%%%%
\chapter{Az OSGi keretrendszer}\label{ch:OSGI}

\begin{osszefoglal}
	E fejezet célja bemutatni az OSGi keretrendszert, illetve annak architektúráját. Bemutatja, hogy a RoboRun projekt miért használja az OSGi keretrendszert. Végül egy általános leírást ad arról, hogy a RoboRun projekt, hogyan használja az OSGi keretrendszert a megerősítéses tanulási kísérletek futtatására és tesztelésére.
\end{osszefoglal}


%%%%%%%%%%%%%%%%%%%%%%%%%%%%%%%%%%%%%%%%%%%%%%%%%%%%%%%%%%%%%%%%%%%%%%%
\section{Az OSGi keretrendszer}\label{sec:OSGI_keretrendszer}
Az OSGi -t eredetileg arra fejlesztették ki, hogy home gateway - ként működjön. Ez azt jelenti, hogy a home gateway kapcsolatban áll egy szolgáltatóval és a felhasználók által kifizetett szolgáltatásokhoz biztosít elérést. Tehát a szolgáltató kezében  van a teljes menedzselés joga, a felhasználó csak használja az adott szolgáltatásokat. 
	
Az OSGi keretrendszer alapötlete a szolgáltatás orientált architektúrára\cite{szolg} vezethető vissza. A szolgáltatás orientált architektúra olyan szolgáltatásokat és komponenseket biztosít, amelyek eleget tesznek egy bizonyos szabványnak, biztonságosak és egymáshoz lazán kapcsolódnak. Ezen komponensek folyamatosan változtathatóak és újra felhasználhatóak.

Az OSGi keretrendszer egy olyan keretrendszer, mely a Java nyelv fölött fut. Az OSGi jelentése, Open Service Gateway Initiative. E keretrendszer célja bővíthető Java alkalmazások fejlesztésének a támogatása. Teljesen dinamikus környezetet biztosít, hiszen képes kezelni a csomagok futás idejű megjelenését és eltűnését a nélkül, hogy a felhasználó bármit is észrevenne ebből. Lehetőséget nyújt különböző szolgáltatások definiálására, amelyek folyamatosan bővíthetőek, változtathatóak, szintén futás időben. Mindezek mellett nagyon jól megvalósítja a komponensek egymástól való elkülönítését. Többféle implementációja ismert az OSGi keretrendszernek, például az Apache Felix\cite{apache} vagy az Eclipse keretrendszer alapjául szolgáló Eclipse Equinox\cite{equinox}.
	   

%%%%%%%%%%%%%%%%%%%%%%%%%%%%%%%%%%%%%%%%%%%%%%%%%%%%%%%%%%%%%%%%%%%%%%%
\section{Az OSGi architektúra}\label{sec:OSGI_architektura}

Az OSGi keretrendszer különböző eszközöket biztosít a szolgáltatások építése érdekében. Ilyen alap eszközök például a batyuk\cite{batyu}.

\begin{figure}[t]
  \centering
  \pgfimage[width=1\linewidth]{images/osgiArchitecture}
  \caption[OSGi architektura]%
  {OSGi architektúra\\
  {\white .}\hfill\url{http://www.osgi.org/Technology/WhatIsOSGi}}
  \label{fig:osgiArchitektura}
\end{figure}

Bundles(batyuk vagy kötegek)

A batyukat az OSGi keretrendszer alapjának tekinthetjük. Általánosan három részből tevődnek össze: Java - kód, statikus erőforrások(pl.: képek) illetve leíró állomány vagy MANIFEST.MF - fájl. A programegységek az OSGi keretrendszerben batyuként kerülnek telepítésre. A batyuk rendelkeznek néhány fontos tulajdonsággal, ilyen tulajdonságok, hogy minden batyuhoz megadhatóak különböző jogok, a batyuk életciklusainak változásai különböző eseményeket generálnak, melyre feliratkozhatnak más batyuk, a batyuk lehetnek futtathatóak, amennyiben implementálják a \texttt{BundleActivator} osztályt, viszont ez nem kötelező. E mellett a batyuk egy nagyon fontos tulajdonsága az, hogy képesek szervizeket regisztrálni, amelyek által más batyuk számára elérhetővé vállnak.

A leíró állomány által értelmezhető a batyu tartalma:
\lstinputlisting{progfiles/MANIFEST.MF}
 
A \textbf{Bundle-Name}- től a \textbf{Bundle-Vendor}- ig a batyuról tárolt információk találhatóak, a \textbf{Bundle-Activator}- azt az osztályt tartalmazza, amelyik elindul a batyu telepítésekor és annak törlésekor leáll. Az \textbf{Export-Package} a batyu által közzétett csomagokat tartalmazza, míg az \textbf{Import-Package} azon csomagokat tartalmazza, amelyekre a batyunak szüksége van a futás során. Természetesen a MANIFEST.MF - állomány más elemeket is tartalmazhat, illetve a példában lévők sem kötelezőek mint. Például a \textbf{Bundle-Activator} címkét nem kötelező megadni, hiszen nem minden batyunak van szüksége \texttt{Activator} osztályra. \\Példa Activator osztályra:
\
\lstset{language=Java}
\lstinputlisting{progfiles/Activator.java}

A batyu telepítésekor az OSGi keretrendszer példányosítja az \texttt{Activator} osztályt és meghívja a start() metódusát automatikusan. A start() metódus megkap egy \texttt{BundleContext}- re mutató referenciát mely által új szervizeket lehet regisztrálni és lekérdezni, a keretrendszer különböző eseményeire lehet feliratkozni, batyukat lehet lekérdezni.

A batyuk rendelkeznek a \texttt{MANIFEST.MF} állomány révén az export - import mechanizmussal. Ez által a batyuk közzétehetik az osztályaikat más batyuk számára. Alapértelmezetten minden batyuban lévő csomag rejtett a többi batyu elől. Azokat a csomagokat amelyeket közzé szeretnénk tenni más batyuk számára az \textbf{Export-Package} címkével tehetjük meg és az \textbf{Import-Package} címke segítségével kérhetjük le azon csomagokat amelyekre szükségünk van más batyukból, természetesen csak akkor, ha ezek publikussá vannak téve a batyu által. A batyuk esetében a csomagfüggőség mellett beszélhetünk batyufüggőségről(Require-Bundle) is. Ezt akkor használják, amennyiben szükség a függőséget csak a teljes batyu képes kielégíteni.

Egy batyuban négyféle csomag érhető el:
\\A batyu által létrehozott csomagok
\\Az \textbf{Import-Package} által megadott csomagok
\\A \textbf{Require-Bundle} által megadott batyu összes publikus csomagja
\\A Java összes függvénykönyvtára
\\
\begin{figure}[h]
  \centering
  \pgfimage[width=1\linewidth]{images/batyuEleres}
  \caption[Batyuk elerese]%
  {Batyu által elérhető csomagok\\
  {\white .}\hfill\url{}}
  \label{fig:BatyukElerese}
\end{figure}
 A batyuknak vannak különböző állapotaik, mely által meghatározható, hogy éppen mi történik velük. Ezen állapotok végig kísérik a batyut a telepítés pillanatától egészen a törlésükig. Ezen állapotokat a batyu életciklusának is szokták nevezni.
\\
\begin{figure}[h]
  \centering
  \pgfimage[width=1\linewidth]{images/batyuLifeCycle}
  \caption[Batyuk eletciklusa]%
  {Batyu életciklusa\\
  {\white .}\hfill\url{https://osgi.org/download/r6/osgi.core-6.0.0.pdf}}
  \label{fig:BatyukEletciklusa}
\end{figure}
\\\textbf{Istalled} vagy \textbf{Installált} állapot: A batyu sikeresen installálásra került az OSGi keretrendszerben
\\\textbf{Resolved} vagy \textbf{Feloldott} állapot: A batyu export illetve import függőségei sikeresen ki vannak elégítve és az általa kiajánlott csomagok is használhatóak a többi batyu számára
\\\textbf{Starting} vagy \textbf{Indulás} állapot: A batyu \texttt{Activator} osztályának start() metódusa meghívásra került de még nem tért vissza
\\\textbf{Active} vagy \textbf{Aktív} állapot: A batyu teljesen aktív a konténerben és használható.
\\\textbf{Stopping} vagy \textbf{Leállás} állapot: A batyu stop() metódusa meghívásra került de még nem tért vissza
\\\textbf{Uninstalled} vagy \textbf{Törölt} állapot: A batyu törlésre került és csak akkor használható újra ha újra telepítik a rendszerbe
\\

\subsection{OSGi szolgáltatások}

Az OSGi architektúra egy másik nagyon fontos építő eleme a szolgáltatások. A szolgáltatások által kódrészleteket lehet elérhetővé tenni, illetve ez biztosítja a batyuk közti dinamikus kommunikációt. Jól definiálja az együttműködés modellt:"publish-find-bind". A szolgáltatás egy közönséges java objektum, amely támogatja a dinamikus, futásidejű változásokat. Ez azt jelenti, hogy futási időben jelenhetnek meg szolgáltatások, melyeket azonnal használatba lehet venni, illetve ezek törlésre is kerülhetnek, szintén futási időben. A szolgáltatások elérése érdekében az OSGi konténer egy szolgáltatás tárolót biztosít. Minden szolgáltatást ide kell beregisztrálni és majd innen lehet kikérni. Minden szolgáltatás kötelező módon egy interfészt implementál és ezen interfész nevén kell regisztrálva legyen. Fontos kiemelni azt, hogy az interfész és az interfész implementációja nem kell egy batyuban legyenek. Az interfészt tartalmazó batyu közzéteszi a megfelelő csomagot, majd ezt a csomagot importálja a szolgáltatás implementációt tartalmazó batyu. Ez biztonság szempontjából is igen fontos tulajdonság lehet. A másik fontos előnye ennek az architektúrának az, hogy így több szolgáltatás is implementálhatja ugyanazt az interfészt. Abban az esetben, ha több szolgáltatás implementálja ugyanazt az interfészt akkor használni kell egy egyedi azonosítót. A szolgáltatások regisztrálását a szolgáltatás tárolóba a \texttt{ServiceRegistration} komponens által lehet megvalósítani.

\lstset{language=Java}
\lstinputlisting{progfiles/ActivatorService.java}

A fenti példa esetén az \texttt{Activator} osztály implementálja a \texttt{BundleActivator} interfészt, mely két metódussal rendelkezik, a \texttt{start(BundleContext context)} és a \texttt{stop(BundleContext context)} metódusokkal. A \texttt{BundleContext}- által új szolgáltatásokat lehet regisztrálni és lekérdezni. A \texttt{context.registerService(Example.class.getName(), new ExampleImpl(), null))} metódus az \texttt{Example} interfész neve által beregisztrálja az OSGi szolgáltatás tárolójába az \texttt{ExampleImpl} szolgáltatást. Az \texttt{ExampleImpl} osztály implementálja az \texttt{Example} interfészt.
\\Abban az esetben ha több szolgáltatás implementálja ugyanazt az interfészt, szükség van az egyedi azonosító használatára.
\lstset{language=Java}
\lstinputlisting{progfiles/ActivatorServiceDictionary.java}
Megadható a \texttt{registerService()} metódusnak egy \texttt{dictionary} paraméter amely, kulcs-érték párokat kell tartalmazzon. Így a szolgáltatás lekérésekor a következő módon hivatkozhatunk a szükséges szolgáltatásra:
\lstset{language=Java}
\lstinputlisting{progfiles/GetService.java}
A szolgáltatások egy másik fontos tulajdonsága az, hogy megőrzik az állapotukat. Amennyiben lekérésre kerül egy szolgáltatás egy batyu által, amely használja is a szolgáltatás metódusait, aztán ugyanezen szolgáltatás ismét lekérésre kerül egy másik batyu által, amely szintén szeretné használni a szolgáltatás metódusait, ő már az előző batyu által beállított értékekkel fog találkozni. Ennek a megoldására az OSGi keretrendszer definiál egy \texttt{ServiceFactory} interfészt, amely két metódussal rendelkezik:
\lstset{language=Java}
\lstinputlisting{progfiles/ServiceFactory.java}
Az \texttt{ExampleServiceFactory} osztály mindig egy új példányát adja vissza az \texttt{ExampleImpl} szolgáltatásnak. Ahhoz, hogy használható legyen az \texttt{ExampleServiceFactory} osztály annyi módosításra van szükség a fenti példához képest, hogy a \texttt{context.registerService()} metódus, nem az \texttt{ExampleImpl} osztály egy példányát fogja paraméterként megkapni, hanem az \texttt{ExampleServiceFactory} osztály egy példányát.
\lstset{language=Java}
\lstinputlisting{progfiles/RegisterServiceWithServiceFactory.java}
A szolgáltatások közzététele megtörténhet a batyu indulásakor, illetve futási időben is.
%!TEX root = dolgozat.tex
%%%%%%%%%%%%%%%%%%%%%%%%%%%%%%%%%%%%%%%%%%%%%%%%%%%%%%%%%%%%
\chapter{A rendszer felépítése és használata}\label{ch:FELEPITES}

\begin{osszefoglal}
	E fejezet célja részletesen ismertetni a rendszer teljes architektúrájának felépítését és a fejlesztés során felmerülő problémák is kiemelésre kerülnek. A fejezet második részében a rendszer használatának ismertetése található. 
\end{osszefoglal}

%%%%%%%%%%%%%%%%%%%%%%%%%%%%%%%%%%%%%%%%%%%%%%%%%%%%%%%%%%%%%%%%%

\section{Nem tudom}\label{sec:Igen}

	
%%%%%%%%%%%%%%%%%%%%%%%%%%%%%%%%%%%%%%%%%%%%%%%%%%%%%%%%%%%%%%%%%%%%%%%%%%%
ENNEK MÉG NEM FOGTAM NEKI! 
\\-SZERETNÉM RÉSZLETEZNI AZ EGÉSZ ARCHITEKTÚRÁT, VALAMI DIAGRAMOKAT.
\\- LEÍRNI AZ EGÉSZ RENDSZER MŰKÖDÉSÉT, KI- KIT HÍV MEG, HOGY STB.
\\- AZ EDDIG KIMARADT KOMPONENSEKET IS MEGEMLÍTENI, MINT A PACKAGES - AMIBEN VANNAK AZ ILYEN TASKSPEC MEG ACTION IMPLEMENTÁCIÓK ÉS AZ AGENTENVIRONMENTLIST KOMPONENSRŐL IS BESZÉLNÉK AMI A LISTÁKAT TARTALMAZZA, AMELYET A WEBES FELÜLET KIÍR.
\\- ARRÓL, HOGY AZ ADATBÁZISBA MI KERÜL BE. ESETLEG, HOGY HOGYAN?!
\\- WEBES FELÜLET MEGVALÓSÍTÁS RÉSZLETEI + HASZNÁLAT
\\
\\AMENNYIBEN A TÖBBI RÉSZ RENDBEN VAN SZERETNÉM EZZEL A FEJEZETTEL KITÖLTENI A HIÁNYZÓ ~8 - 10 oldalt.
%%%%%%%%%%%%%%%%%%%%%%%%%%%%%%%%%%%%%%%%%%%%%%%%%%%%%%%%%%%%%%%%%%%%%%%%%%%






%!TEX root = dolgozat.tex
%%%%%%%%%%%%%%%%%%%%%%%%%%%%%%%%%%%%%%%%%%%%%%%%%%%%%%%%%%%%%%%%%%%%%%%
\chapter{Következtetés és továbbfejlesztési lehetőségek}\label{ch:KOVETKEZTETESEK}

A RoboRun projekt keretein belül megvalósításra került egy megerősítéses tanulási algoritmusok tesztelésére szolgáló teljes rendszer, amely lehetőséget biztosít a folyamatos és könnyed használatra. A rendszer dinamikus, megvalósítja a komponensek egymástól való elválasztását.A projekthez tartozik egy webes felület, melyen megtekinthetőek a telepített batyuk, az aktív tesztek, illetve a már lefuttatott tesztek eredményei. A rendszer ezen eredményeket egy adatbázisban tárolja. A RoboRun projekt teljesen az OSGi alkalmazás modellre épül.

A RoboRun projekt architektúrája tervezésekor óriási hangsúly volt fektetve a továbbfejleszthetőségre, így a projekt felépítése is ezt tükrözi.

Az OSGi alkalmazás modell által minden komponens külön van választva, így a már meglévő részek rugalmasan továbbfejleszthetőek és kiegészíthetőek. A RoboRun projekt jelen formájában egy OSGi konténerben van telepítve, mely egy GlassFish szerveren fut.

A projekt továbbfejlesztésére számos lehetőség létezik. Egyik legfontosabb ilyen lehetőség, a projekthez egy Eclipse Plugin\cite{eclipseplugin} készítése, mely által az Eclipse fejlesztői környezet, egy olyan környezetet garantálhat, mely megkönnyíti a megerősítéses tanulási algoritmusok implementálását, illetve ez által megvalósítható az, hogy az \texttt{Experiment} komponens a saját gépről futtatható legyen, míg a rendszer többi része egy központi elérésű szerveren található. Ehhez szükség van felhasználni az OSGi által biztosított OSGi Remote Services tulajdonságot. Ez azt jelenti, hogy az OSGi képes távoli metódushívásokra, oly módon, hogy a rendszer egyik része fut egy OSGi konténerben, míg a rendszer többi része  egy teljesen más OSGi konténerben található. Az Eclipse Plugin esetén az Eclipse fejlesztői környezet lenne az egyik OSGi konténer, míg a másik a GlassFish szerverre telepített lenne. Így az erőforrás igényes komponensek, mint például a \texttt{Agent}, \texttt{Environment} és a \texttt{RoboCommnication} komponensek a GlassFish szerveren futnának és az \texttt{Experiment} komponens futna az Eclipse fejlesztői környezetből, amely egy saját lokális számítógépen található. 
Egy másik továbbfejlesztési lehetőségként érdemes megemlíteni az RMI(Remote Method Invocation) kommunikációt. Mely kapcsolatot létesít a szerverrel és képes lekérdezni különböző objektumokat. Ezzel az a probléma, hogy a lekérdezett objektumok szerializálásra kerülnek elküldés előtt és az OSGi szolgáltatásokat nem lehet szerializálni a konténeren kívülre, mert ezen szolgáltatások csak a konténeren belül elérhetőek. Amennyiben erre a problémára sikerülne találni megoldást, ez a továbbfejlesztés nagyon egyszerű és hasznos módja lenne, hiszen ezáltal szintén teljesen áthelyezhető az \texttt{Experiment} komponens saját lokális számítógépre.

A rendszer jelenleg rendelkezik néhány alap környezettel és két tanuló algoritmussal. Viszont ezek implementálása nem a projekt része. Fontos továbbfejlesztési lehetőség lehet, újabb környezetek és tanulási algoritmusokkal kiegészíteni a jelenlegi rendszert, melyek hozzáadása a rendszer tervezésének és felépítésének köszönhetően egyszerűen eszközölhető. 

A rendszer továbbfejlesztését a webes felülettel lehetne folytatni, amely jelen formájában egy prototípus és rendelkezik a legfontosabb alapfunkcionalitásokkal, melyek szükségesek ahhoz, hogy információkat kapjunk a rendszerről és a rendszer által futtatott tesztek állapotairól, illetve az ezek által generált adatok halmazáról. Ezen funkcionalitások könnyedén kibővíthetőek a \ref{subsec:WebesFelulet} alfejezetben leírt felépítése által.

A dolgozat\cite{dolgozat} és a forráskód\cite{forras} elérhető GitHub-on. 

%%%%%%%%%%%%%%%%%%%%%%%%%%%%%%%%%%%%%%%%%%%%%%%%%%%%%%%%%%%%%%%%%%%%%%%%%%%



\appendix

{ \renewcommand{\baselinestretch}{1.5}\normalsize %
  \setlength{\itemsep}{-2.4mm}
  \setlength{\bibspacing}{0.67\baselineskip}
  \bibliographystyle{abbrvnat_hu}
  \bibliography{hivatkozasok}
}

\end{document}
