%!TEX root = dolgozat.tex
%%%%%%%%%%%%%%%%%%%%%%%%%%%%%%%%%%%%%%%%%%%%%%%%%%%%%%%%%%%%%%%%%%%%%%%
\chapter{Az OSGi keretrendszer}\label{ch:MAT}

\begin{osszefoglal}
	E fejezet célja bemutatni az OSGi keretrendszert, illetve annak architektúráját. Bemutatja, hogy a RoboRun projekt miért használja az OSGi keretrendszert. Végül egy általános leírást ad arról, hogy a RoboRun projekt, hogyan használja az OSGi keretrendszert a megerősítéses tanulási kísérletek futtatására és tesztelésére.
\end{osszefoglal}


%%%%%%%%%%%%%%%%%%%%%%%%%%%%%%%%%%%%%%%%%%%%%%%%%%%%%%%%%%%%%%%%%%%%%%%
\section{Az OSGi keretrendszer}\label{sec:MAT:bev}
Az OSGi -t eredetileg arra fejlesztették ki, hogy home gateway - ként működjön. Ez azt jelenti, hogy a home gateway kapcsolatban áll egy szolgáltatóval és a felhasználók által kifizetett szolgáltatásokhoz biztosít elérést. Tehát a szolgáltató kezében  van a teljes menedzselés joga, a felhasználó csak használja az adott szolgáltatásokat. 

	Az OSGi keretrendszer egy olyan keretrendszer, mely a Java nyelv fölött fut. Az OSGi jelentése, Open Service Gateway Initiative. E keretrendszer célja bővíthető Java alkalmazások fejlesztésének a támogatása. Teljesen dinamikus környezetet biztosít, hiszen képes kezelni a csomagok futás idejű megjelenését és eltűnését a nélkül, hogy a felhasználó bármit is észrevenne ebből. Lehetőséget nyújt különböző szolgáltatások definiálására, amelyek folyamatosan bővíthetőek, változtathatóak, szintén futás időben. Mindezek mellett nagyon jól megvalósítja a komponensek egymástól való elkülönítését. 
	
	Az OSGi biztosít néhány nagyon fontos és nélkülözhetetlen eszközt, amelyek segítségével különböző szolgáltatásokat lehet építeni.   


%%%%%%%%%%%%%%%%%%%%%%%%%%%%%%%%%%%%%%%%%%%%%%%%%%%%%%%%%%%%%%%%%%%%%%%
\section{Az OSGi architektúra}\label{sec:MAT:muv}




