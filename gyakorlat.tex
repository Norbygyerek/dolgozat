%!TEX root = dolgozat.tex
%%%%%%%%%%%%%%%%%%%%%%%%%%%%%%%%%%%%%%%%%%%%%%%%%%%%%%%%%%%%%%%%%%%%%%%
\chapter{Az OSGi keretrendszer}\label{ch:MAT}

\begin{osszefoglal}
	E fejezet célja bemutatni az OSGi keretrendszert, illetve annak architektúráját. Bemutatja, hogy a RoboRun projekt miért használja az OSGi keretrendszert. Végül egy általános leírást ad arról, hogy a RoboRun projekt, hogyan használja az OSGi keretrendszert a megerősítéses tanulási kísérletek futtatására és tesztelésére.
\end{osszefoglal}


%%%%%%%%%%%%%%%%%%%%%%%%%%%%%%%%%%%%%%%%%%%%%%%%%%%%%%%%%%%%%%%%%%%%%%%
\section{Az OSGi keretrendszer}\label{sec:MAT:bev}
Az OSGi -t eredetileg arra fejlesztették ki, hogy home gateway - ként működjön. Ez azt jelenti, hogy a home gateway kapcsolatban áll egy szolgáltatóval és a felhasználók által kifizetett szolgáltatásokhoz biztosít elérést. Tehát a szolgáltató kezében  van a teljes menedzselés joga, a felhasználó csak használja az adott szolgáltatásokat. 
	
Az OSGi keretrendszer alapötlete a szolgáltatásorientált architektúrára[3] vezethető vissza. A szolgáltatásorientált architektúra olyan szolgáltatásokat és komponenseket biztosít, amelyek eleget tesznek egy bizonyos szabványnak, biztonságosak és egymáshoz lazán kapcsolódnak. Ezen komponensek folyamatosan változtathatóak és újra felhasználhatóak.

Az OSGi keretrendszer egy olyan keretrendszer, mely a Java nyelv fölött fut. Az OSGi jelentése, Open Service Gateway Initiative. E keretrendszer célja bővíthető Java alkalmazások fejlesztésének a támogatása. Teljesen dinamikus környezetet biztosít, hiszen képes kezelni a csomagok futás idejű megjelenését és eltűnését a nélkül, hogy a felhasználó bármit is észrevenne ebből. Lehetőséget nyújt különböző szolgáltatások definiálására, amelyek folyamatosan bővíthetőek, változtathatóak, szintén futás időben. Mindezek mellett nagyon jól megvalósítja a komponensek egymástól való elkülönítését. Többféle implementációja ismert az OSGi keretrendszernek, például az Apache Felix vagy az Eclipse keretrendszer alapjául szolgáló Eclipse Equinox.
	   

%%%%%%%%%%%%%%%%%%%%%%%%%%%%%%%%%%%%%%%%%%%%%%%%%%%%%%%%%%%%%%%%%%%%%%%
\section{Az OSGi architektúra}\label{sec:MAT:muv}

Az OSGi keretrendszer különböző eszközöket biztosít a szolgáltatások építése érdekében. Ilyen alap eszközök például a batyuk.

\begin{figure}[t]
  \centering
  \pgfimage[width=1\linewidth]{images/osgiArchitecture}
  \caption[OSGi architektura]%
  {OSGi architektúra\\
  {\white .}\hfill\url{http://www.osgi.org/Technology/WhatIsOSGi}}
  \label{fig:ALAP:sm1}
\end{figure}

Bundles(batyuk vagy kötegek)

A batyukat az OSGi keretrendszer alapjának tekinthetjük. Általánosan három részből tevődnek össze: Java - kód, statikus erőforrások(pl.: képek) illetve leíró állomány vagy MANIFEST.MF - fájl. A programegységek az OSGi keretrendszerben batyuként kerülnek telepítésre. A batyuk rendelkeznek néhány fontos tulajdonsággal, ilyen tulajdonságok, hogy minden batyuhoz megadhatóak különböző jogok, a batyuk életciklusainak változásai különböző eseményeket generálnak, melyre feliratkozhatnak más batyuk, a batyuk lehetnek futtathatóak, amennyiben implementálják a \texttt{BundleActivator} osztályt, viszont ez nem kötelező. E mellett a batyuk egy nagyon fontos tulajdonsága az, hogy képesek szervizeket regisztrálni, amelyek által más batyuk számára elérhetővé vállnak.

A leíró állomány által értelmezhető a batyu tartalma:
\lstinputlisting{progfiles/MANIFEST.MF}
 
A \textbf{Bundle-Name}- től a \textbf{Bundle-Vendor}- ig a batyuról tárolt információk találhatóak, a \textbf{Bundle-Activator}- azt az osztályt tartalmazza, amelyik elindul a batyu telepítésekor és annak törlésekor leáll. Az \textbf{Export-Package} a batyu által közzétett csomagokat tartalmazza, míg az \textbf{Import-Package} azon csomagokat tartalmazza, amelyekre a batyunak szüksége van a futás során. Természetesen a MANIFEST.MF - állomány más elemeket is tartalmazhat, illetve a példában lévők sem kötelezőek mint. Például a \textbf{Bundle-Activator} címkét nem kötelező megadni, hiszen nem minden batyunak van szüksége \texttt{Activator} osztályra. \\Példa Activator osztályra:
\
\lstset{language=Java}
\lstinputlisting{progfiles/Activator.java}

A batyu telepítésekor az OSGi keretrendszer példányosítja az \texttt{Activator} osztályt és meghívja a start() metódusát automatikusan. A start() metódus megkap egy \texttt{BundleContext}- re mutató referenciát mely által új szervizeket lehet regisztrálni és lekérdezni, a keretrendszer különböző eseményeire lehet feliratkozni, batyukat lehet lekérdezni.

A batyuk rendelkeznek a \texttt{MANIFEST.MF} állomány révén az export - import mechanizmussal. Ez által a batyuk közzétehetik az osztályaikat más batyuk számára. Alapértelmezetten minden batyuban lévő csomag rejtett a többi batyu elől. Azokat a csomagokat amelyeket közzé szeretnénk tenni más batyuk számára az \textbf{Export-Package} címkével tehetjük meg és az \textbf{Import-Package} címke segítségével kérhetjük le azon csomagokat amelyekre szükségünk van más batyukból, természetesen csak akkor, ha ezek publikussá vannak téve a batyu által. A batyuk esetében a csomagfüggőség mellett beszélhetünk batyufüggőségről(Require-Bundle) is. Ezt akkor használják, amennyiben szükség a függőséget csak a teljes batyu képes kielégíteni.

Egy batyuban négyféle csomag érhető el:
\\A batyu által létrehozott csomagok
\\Az \textbf{Import-Package} által megadott csomagok
\\A \textbf{Require-Bundle} által megadott batyu összes publikus csomagja
\\A Java összes függvénykönyvtára
